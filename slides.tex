% Start a new section (text is displayed on top of a frame)
\section{Section title}

% Frame with items
\begin{frame}{Frame title}
  \begin{itemize}
    \item First level
    \begin{itemize}
      \item Second level
      \begin{itemize}
      \item Third level
    \end{itemize}
  \end{itemize}
\end{itemize}
\end{frame}

% Frame with plain text
\begin{frame}{Lorem ipsum dolor sit amet}
Lorem ipsum dolor sit amet, consectetuer adipiscing elit. Aenean commodo ligula eget dolor. Aenean massa. Cum sociis natoque penatibus et magnis dis parturient montes, nascetur ridiculus mus. Donec quam felis, ultricies nec, pellentesque eu, pretium quis, sem. Nulla consequat massa quis enim. Donec pede justo, fringilla vel, aliquet nec, vulputate eget, arcu. In enim justo, rhoncus ut, imperdiet a, venenatis vitae, justo. Nullam dictum felis eu pede mollis pretium. Integer tincidunt. Cras dapibus. Vivamus elementum semper nisi. Aenean vulputate eleifend tellus. Aenean leo ligula, porttitor eu, consequat vitae, eleifend ac, enim. Aliquam lorem ante, dapibus in, viverra quis, feugiat a, tellus. Phasellus viverra nulla ut metus varius laoreet. Quisque rutrum. Aenean imperdiet. Etiam ultricies nisi vel augue. Curabitur ullamcorper ultricies nisi. Nam eget dui. Etiam rhoncus. Maecenas tempus, tellus eget condimentum rhoncus, sem quam semper libero, sit amet adipiscing sem neque sed ipsum.
\end{frame}

% Frame with text and picture
\begin{frame}{Frame title}
\begin{columns}[onlytextwidth]\
  % Text on the left
  \begin{column}{.5\textwidth}
    \begin{itemize}
      \item First level
      \begin{itemize}
        \item Second level
        \begin{itemize}
          \item Third level
        \end{itemize}
      \end{itemize}
    \end{itemize}
  \end{column}
  % Picture on the right
  \begin{column}{.5\textwidth}
    \hfill\raisebox{-10cm}[0pt][10cm]{\includegraphics[height=10cm]{figure.jpg}}
  \end{column}
\end{columns}
\end{frame}

% Frame with large picture and caption
\begin{frame}
\begin{figure}
\includegraphics[width=\textwidth]{title_large}
\caption{Image description}
\end{figure}
\end{frame}

% New section (with charts)
% Note: Using newline in a section is not recommended, this is
% just an example
\section{Section title\newline
Example of a two-line title}

% Frame with horizontal bar chart
\begin{frame}{Add title}
\begin{center}
\begin{tikzpicture}
  \begin{axis}[hor_barchart,symbolic y coords={Kategorie 1,Kategorie 2,Kategorie 3, Kategorie 4},%
               width=0.7\textwidth,height=10cm]
    \addplot [draw opacity=0,fill=rwth-50] coordinates {(4.5,Kategorie 4) (3.5,Kategorie 3) (2.5,Kategorie 2) (4.3,Kategorie 1)};
    \addplot [draw opacity=0,fill=rwth] coordinates {(2.8,Kategorie 4) (1.8,Kategorie 3) (4.4,Kategorie 2) (2.4,Kategorie 1)};
    \legend{Datenreihe 1,Datenreihe 2}
  \end{axis}
\end{tikzpicture}
\end{center}
\end{frame}

% Frame with vertical bar chart
\begin{frame}{Add title}
\begin{center}
\begin{tikzpicture}
  \begin{axis}[ver_barchart,symbolic x coords={Kategorie 1,Kategorie 2,Kategorie 3, Kategorie 4},%
               width=0.7\textwidth,height=10cm]
    \addplot [draw opacity=0,fill=rwth] coordinates {(Kategorie 4,4.5) (Kategorie 3,3.5) (Kategorie 2,2.5) (Kategorie 1,4.3)};
    \addplot [draw opacity=0,fill=rwth-75] coordinates {(Kategorie 4,2.8) (Kategorie 3,1.8) (Kategorie 2,4.4) (Kategorie 1,2.4)};
    \addplot [draw opacity=0,fill=rwth-50] coordinates {(Kategorie 4,3) (Kategorie 3,0.5) (Kategorie 2,5.4) (Kategorie 1,3.4)};
    \addplot [draw opacity=0,fill=rwth-25] coordinates {(Kategorie 4,3.5) (Kategorie 3,1.5) (Kategorie 2,2.2) (Kategorie 1,4.2)};
  \legend{Datenreihe 1,Datenreihe 2,Datenreihe 3,Datenreihe 4}
  \end{axis}
\end{tikzpicture}
\end{center}
\end{frame}

% Frame with pie chart
\begin{frame}{Add title}
\begin{center}
\hspace{2cm}
\begin{tikzpicture}
[
  pie chart,
  slice type={one}{rwth-25},
  slice type={two}{rwth-50},
  slice type={three}{rwth-75},
  slice type={four}{rwth},
  pie values/.style={font={\Large}},
  scale=5
]
  \pie{}{29/one,17/two,23.6/three,30.4/four}
  \legend[xshift=1.5cm,yshift=0.8cm]{{Kategorie 1}/one, {Kategorie 2}/two, {Kategorie 3}/three, {Kategorie 4}/four}
\end{tikzpicture}
\end{center}
\end{frame}

% Final frame, subtext is optional
% Note: should be plain
\setbeamertemplate{final page}[rwth][Any questions?]{Thank you for your attention}
\begin{frame}[plain]
\usebeamertemplate{final page}
\end{frame}